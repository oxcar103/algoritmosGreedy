%%%%
% Modificación de una plantilla de Latex para adaptarla al castellano.
%%%

%%%%%%%%%%%%%%%%%%%%%%%%%%%%%%%%%%%%%%%%%
% Thin Sectioned Essay
% LaTeX Template
% Version 1.0 (3/8/13)
%
% This template has been downloaded from:
% http://www.LaTeXTemplates.com
%
% Original Author:
% Nicolas Diaz (nsdiaz@uc.cl) with extensive modifications by:
% Vel (vel@latextemplates.com)
%
% License:
% CC BY-NC-SA 3.0 (http://creativecommons.org/licenses/by-nc-sa/3.0/)
%
%%%%%%%%%%%%%%%%%%%%%%%%%%%%%%%%%%%%%%%%%

%----------------------------------------------------------------------------------------
%	PACKAGES AND OTHER DOCUMENT CONFIGURATIONS
%----------------------------------------------------------------------------------------

\documentclass[a4paper, 11pt]{article} % Font size (can be 10pt, 11pt or 12pt) and paper size (remove a4paper for US letter paper)

\usepackage[protrusion=true,expansion=true]{microtype} % Better typography
\usepackage{graphicx} % Required for including pictures
\usepackage[usenames,dvipsnames]{color} % Coloring code
\usepackage{wrapfig} % Allows in-line images
\usepackage[utf8]{inputenc}

% sudo apt-get install texlive-lang-spanish
\usepackage[spanish]{babel} % English language/hyphenation
\selectlanguage{spanish}
% Hay que pelearse con babel-spanish para el alineamiento del punto decimal
\decimalpoint
\usepackage{dcolumn}
\newcolumntype{d}[1]{D{.}{\esperiod}{#1}}
\makeatletter
\addto\shorthandsspanish{\let\esperiod\es@period@code}
\makeatother

\usepackage{longtable}
\usepackage{tabu}
\usepackage{supertabular}

\usepackage{multicol}
\newsavebox\ltmcbox

% Para algoritmos
\usepackage{algorithm}
\usepackage{algorithmic}
\usepackage{amsthm}
\floatname{algorithm}{Algoritmo}
\renewcommand{\listalgorithmname}{Lista de algoritmos}
\renewcommand{\algorithmicrequire}{\textbf{Entrada:}}
\renewcommand{\algorithmicensure}{\textbf{Salida:}}
\renewcommand{\algorithmicend}{\textbf{fin}}
\renewcommand{\algorithmicif}{\textbf{si}}
\renewcommand{\algorithmicthen}{\textbf{entonces}}
\renewcommand{\algorithmicelse}{\textbf{en otro caso}}
\renewcommand{\algorithmicelsif}{\algorithmicelse,\ \algorithmicif}
\renewcommand{\algorithmicendif}{\algorithmicend\ \algorithmicif}
\renewcommand{\algorithmicfor}{\textbf{para }}
\renewcommand{\algorithmicforall}{\textbf{para cada}}
\renewcommand{\algorithmicdo}{\textbf{}}
\renewcommand{\algorithmicendfor}{\algorithmicend\ \algorithmicfor}
\renewcommand{\algorithmicwhile}{\textbf{mientras}}
\renewcommand{\algorithmicendwhile}{\algorithmicend\ \algorithmicwhile}
\renewcommand{\algorithmicloop}{\textbf{repetir}}
\renewcommand{\algorithmicendloop}{\algorithmicend\ \algorithmicloop}
\renewcommand{\algorithmicrepeat}{\textbf{repetir}}
\renewcommand{\algorithmicuntil}{\textbf{hasta que}}
\renewcommand{\algorithmicprint}{\textbf{imprimir}} 
\renewcommand{\algorithmicreturn}{\textbf{devolver}} 
\renewcommand{\algorithmictrue}{\textbf{true }} 
\renewcommand{\algorithmicfalse}{\textbf{false }} 
\renewcommand{\algorithmicand}{\textbf{y}}
\renewcommand{\algorithmicor}{\textbf{o}}


\usepackage[section]{placeins} % Para gráficas en su sección.
\usepackage{mathpazo} % Use the Palatino font
\usepackage[T1]{fontenc} % Required for accented characters
\newenvironment{allintypewriter}{\ttfamily}{\par}
\setlength{\parindent}{0pt}
\parskip=8pt
\linespread{1.05} % Change line spacing here, Palatino benefits from a slight increase by default

\makeatletter
\renewcommand\@biblabel[1]{\textbf{#1.}} % Change the square brackets for each bibliography item from '[1]' to '1.'
\renewcommand{\@listI}{\itemsep=0pt} % Reduce the space between items in the itemize and enumerate environments and the bibliography
\newcommand{\imagen}[2]{\begin{center} \includegraphics[width=90mm]{#1} \\#2 \end{center}}

\renewcommand{\maketitle}{ % Customize the title - do not edit title and author name here, see the TITLE block below
\begin{flushright} % Right align
{\LARGE\@title} % Increase the font size of the title

\vspace{50pt} % Some vertical space between the title and author name

{\large\@author} % Author name
\\\@date % Date

\vspace{40pt} % Some vertical space between the author block and abstract
\end{flushright}
}

%----------------------------------------------------------------------------------------
%	TITLE
%----------------------------------------------------------------------------------------

\title{\textbf{Práctica 3}\\ % Title
Algoritmos Greedy} % Subtitle

\author{\textsc{Óscar Bermúdez,\\Francisco David Charte,\\Ignacio Cordón,\\José Carlos Entrena,\\Mario Román} % Author
\\{\textit{Universidad de Granada}}} % Institution

\date{\today} % Date

%----------------------------------------------------------------------------------------

\begin{document}

\maketitle % Print the title section

\renewcommand{\abstractname}{Resumen} % Uncomment to change the name of the abstract to something else
\begin{abstract}
\end{abstract}
{\parskip=2pt
\tableofcontents
}
\pagebreak


\section{Terminales de venta}
\subsection{Implementación}
  Implementamos en Ruby la heurítica pedida.\\
  \small
    \texttt{% Generator: GNU source-highlight, by Lorenzo Bettini, http://www.gnu.org/software/src-highlite
\noindent
\mbox{}\textbf{\textcolor{Blue}{def}}\ cambio\ \textcolor{BrickRed}{(}monedas\textcolor{BrickRed}{,}\ precio\textcolor{BrickRed}{)} \\
\mbox{}\ \ vuelta\ \textcolor{BrickRed}{=}\ \textcolor{BrickRed}{[]} \\
\mbox{} \\
\mbox{}\ \ monedas\textcolor{BrickRed}{.}sort\textcolor{BrickRed}{.}reverse\textcolor{BrickRed}{.}each\ \textcolor{Red}{\{}\ \textcolor{BrickRed}{$|$}moneda\textcolor{BrickRed}{$|$} \\
\mbox{}\ \ \ \ numero$\_$monedas\ \textcolor{BrickRed}{=}\ precio\ \textcolor{BrickRed}{/}\ moneda \\
\mbox{}\ \ \ \ precio\ \textcolor{BrickRed}{=}\ precio\ \textcolor{BrickRed}{-}\ numero$\_$monedas\textcolor{BrickRed}{*}moneda \\
\mbox{} \\
\mbox{}\ \ \ \ vuelta\textcolor{BrickRed}{.}push\ \textcolor{BrickRed}{[}moneda\textcolor{BrickRed}{,}\ numero$\_$monedas\textcolor{BrickRed}{]} \\
\mbox{}\ \ \textcolor{Red}{\}} \\
\mbox{} \\
\mbox{}\ \ \textbf{\textcolor{Blue}{return}}\ vuelta \\
\mbox{}\textbf{\textcolor{Blue}{end}} \\
\mbox{}
}
  \normalsize

  
\section{Red de comunicaciones}
  \subsection{Algoritmo}
    Dado un conjunto de ciudades, buscamos interconectarlas con una red que minimice la longitud de red. Modelizaremos el problema como un grafo completo $G$
    del conjunto de ciudades $E$. Con aristas $V = \left\{ [a,b] |\ a,b \in E \right\}$, donde la arista conectando los nodos $a$ y $b$ tiene peso igual a la distancia que los separa:
    \begin{equation}
     \forall [a,b]\in V:\quad  w([a,b]) = dist (a,b)
    \end{equation}

    Pretendemos interconectar ciudades de manera que la suma total de las distancias de los caminos hechos sea mínima.
    Es decir, buscamos el subgrafo recubridor de menor peso. Como un grafo recubridor con ciclos tiene siempre un subgrafo recubridor estrictamente contenido en él,
    buscamos sólo entre los grafos acíclicos. El subgrafo acíclico recubridor de menor costo es el árbol recubridor minimal.
    
    Son conocidos los algoritmos de Prim y Kruskal para calcular el árbol recubridor minimal. Ambos alcanzan una complejidad temporal de $\mathcal{O}(|E|log|V|)$ usando árboles binarios y
    listas de adyacencia para representar el grafo.
    
  \subsection{Triangulación de Delaunay}
    Para reducir la carga del algoritmo, podemos reducir el grafo sobre el que buscamos el árbol generador minimal. El grafo inicial es completo, ya que toda
    ciudad es suceptible de ser comunicada con cualquiera otra. Por lo tanto, buscamos el árbol generador minimal en todas las aristas posibles.
    
    Sin embargo, podemos demostrar que el árbol generador minimal está contenido en la triangulación de Delaunay; y que, por tanto, sólo es necesario aplicar el algoritmo de Kruskal o Prim
    al subgrafo resultante de la triangulación.
    
    \begin{proof}
    
      Demostraremos que cada arista del árbol generador está contenida en la triangulación de Delaunay.
      Sea $(p,q)$ una arista arbitraria del árbol generador. 
      Consideramos el círculo que tiene como diámetro $\overline{pq}$, si hubiera otro punto $r$ en este círculo, tendríamos:
      \begin{equation}
       \overrightarrow{pr} \leq \overrightarrow{pq} \qquad \overrightarrow{rq} \leq \overrightarrow{pq}
      \end{equation}

      Y entonces podríamos formar el ciclo $p,r,q,p$. Sabemos que la arista de mayor peso en un ciclo no forma parte del árbol generador minimal y por tanto $\overline{pq}$
      no pertenecería a él, llegando a contradicción.

      Así, no puede haber ningún punto en el círculo que tiene como diámetro a $\overline{pq}$. Una arista que cumple esto está forzosamente en el diagrama de Delaunay.
    \end{proof}
    
\section{Segmentación de clientes}

\section{Asignación de trabajos}

\section{Asignación de aulas}

\section{Memorias caché}
    
\section{El problema de asignación cuadrática}
    
\end{document}