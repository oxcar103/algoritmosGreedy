%%%%
% Modificación de una plantilla de Latex para adaptarla al castellano.
%%%

%%%%%%%%%%%%%%%%%%%%%%%%%%%%%%%%%%%%%%%%%
% Thin Sectioned Essay
% LaTeX Template
% Version 1.0 (3/8/13)
%
% This template has been downloaded from:
% http://www.LaTeXTemplates.com
%
% Original Author:
% Nicolas Diaz (nsdiaz@uc.cl) with extensive modifications by:
% Vel (vel@latextemplates.com)
%
% License:
% CC BY-NC-SA 3.0 (http://creativecommons.org/licenses/by-nc-sa/3.0/)
%
%%%%%%%%%%%%%%%%%%%%%%%%%%%%%%%%%%%%%%%%%

%----------------------------------------------------------------------------------------
%	PACKAGES AND OTHER DOCUMENT CONFIGURATIONS
%----------------------------------------------------------------------------------------

\documentclass[a4paper, 11pt]{article} % Font size (can be 10pt, 11pt or 12pt) and paper size (remove a4paper for US letter paper)

\usepackage[protrusion=true,expansion=true]{microtype} % Better typography
\usepackage{graphicx} % Required for including pictures
\usepackage[usenames,dvipsnames]{color} % Coloring code
\usepackage{wrapfig} % Allows in-line images
\usepackage[utf8]{inputenc}

% Imágenes
\usepackage{graphicx} 

\usepackage{amsmath}
% para importar svg
%\usepackage[generate=all]{svgfig}

% sudo apt-get install texlive-lang-spanish
\usepackage[spanish]{babel} % English language/hyphenation
\selectlanguage{spanish}
% Hay que pelearse con babel-spanish para el alineamiento del punto decimal
\decimalpoint
\usepackage{dcolumn}
\newcolumntype{d}[1]{D{.}{\esperiod}{#1}}
\makeatletter
\addto\shorthandsspanish{\let\esperiod\es@period@code}
\makeatother

\usepackage{longtable}
\usepackage{tabu}
\usepackage{supertabular}

\usepackage{multicol}
\newsavebox\ltmcbox

% Para algoritmos
\usepackage{algorithm}
\usepackage{algorithmic}
\usepackage{amsthm}
\floatname{algorithm}{Algoritmo}
\renewcommand{\listalgorithmname}{Lista de algoritmos}
\renewcommand{\algorithmicrequire}{\textbf{Entrada:}}
\renewcommand{\algorithmicensure}{\textbf{Salida:}}
\renewcommand{\algorithmicend}{\textbf{fin}}
\renewcommand{\algorithmicif}{\textbf{si}}
\renewcommand{\algorithmicthen}{\textbf{entonces}}
\renewcommand{\algorithmicelse}{\textbf{en otro caso}}
\renewcommand{\algorithmicelsif}{\algorithmicelse,\ \algorithmicif}
\renewcommand{\algorithmicendif}{\algorithmicend\ \algorithmicif}
\renewcommand{\algorithmicfor}{\textbf{para }}
\renewcommand{\algorithmicforall}{\textbf{para cada}}
\renewcommand{\algorithmicdo}{\textbf{}}
\renewcommand{\algorithmicendfor}{\algorithmicend\ \algorithmicfor}
\renewcommand{\algorithmicwhile}{\textbf{mientras}}
\renewcommand{\algorithmicendwhile}{\algorithmicend\ \algorithmicwhile}
\renewcommand{\algorithmicloop}{\textbf{repetir}}
\renewcommand{\algorithmicendloop}{\algorithmicend\ \algorithmicloop}
\renewcommand{\algorithmicrepeat}{\textbf{repetir}}
\renewcommand{\algorithmicuntil}{\textbf{hasta que}}
\renewcommand{\algorithmicprint}{\textbf{imprimir}} 
\renewcommand{\algorithmicreturn}{\textbf{devolver}} 
\renewcommand{\algorithmictrue}{\textbf{true }} 
\renewcommand{\algorithmicfalse}{\textbf{false }} 
\renewcommand{\algorithmicand}{\textbf{y}}
\renewcommand{\algorithmicor}{\textbf{o}}


% Para matrices
\usepackage{amsmath}

% Símbolos matemáticos
\usepackage{amssymb}
\let\oldemptyset\emptyset
\let\emptyset\varnothing

\usepackage[section]{placeins} % Para gráficas en su sección.
\usepackage{mathpazo} % Use the Palatino font
\usepackage[T1]{fontenc} % Required for accented characters
\newenvironment{allintypewriter}{\ttfamily}{\par}
\setlength{\parindent}{0pt}
\parskip=8pt
\linespread{1.05} % Change line spacing here, Palatino benefits from a slight increase by default

\newcommand{\ef}[1]{$\mathcal{O}#1$}

\makeatletter
\renewcommand\@biblabel[1]{\textbf{#1.}} % Change the square brackets for each bibliography item from '[1]' to '1.'
\renewcommand{\@listI}{\itemsep=0pt} % Reduce the space between items in the itemize and enumerate environments and the bibliography
\newcommand{\imagen}[2]{\begin{center} \includegraphics[width=90mm]{#1} \\#2 \end{center}}

\renewcommand{\maketitle}{ % Customize the title - do not edit title and author name here, see the TITLE block below
\begin{flushright} % Right align
{\LARGE\@title} % Increase the font size of the title

\vspace{50pt} % Some vertical space between the title and author name

{\large\@author} % Author name
\\\@date % Date

\vspace{40pt} % Some vertical space between the author block and abstract
\end{flushright}
}

%----------------------------------------------------------------------------------------
%	TITLE
%----------------------------------------------------------------------------------------

\title{\textbf{Práctica 3}\\ % Title
QAP: Asginación cuadrática} % Subtitle

\author{\textsc{Óscar Bermúdez,\\Francisco David Charte,\\Ignacio Cordón,\\José Carlos Entrena,\\Mario Román} % Author
\\{\textit{Universidad de Granada}}} % Institution

\date{\today} % Date

%----------------------------------------------------------------------------------------

\begin{document}

\maketitle % Print the title section

\renewcommand{\abstractname}{Resumen} % Uncomment to change the name of the abstract to something else
\begin{abstract}
\end{abstract}
{\parskip=2pt
\tableofcontents
}
\pagebreak


\section{QAP}
Dado un conjunto de $n$ instalaciones (p.e. fábricas), con un flujo de transporte y
un conste asociado entre dichas instalaciones; y dadas $n$ posibles localizaciones,
el problema consiste en buscar ubicación a cada una de las instalaciones
de modo que siendo $d(i,j)$ la distancia entre la localización $i$ y la $j$
y siendo $w(i,j)$ el peso asociado al flujo de materiales entre la instalación
$i$ y la $j$, se consiga minimizar:
\begin{equation}
\sum_{i,j} w(i,j) d(p(i),p(j))
\label{coste}
\end{equation}

donde $p()$ define una permutación sobre el conjunto de instalaciones

El código que presentaremos en este ejercicio está escrito en Ruby.
\subsection{Heurística greedy v1}

\begin{algorithm}[H]
	\begin{algorithmic}[1]
		\REQUIRE \ \\
        	\texttt{qap}, instancia del problema \\\
     	\STATE{\texttt{min=qap.cost}, almacena \ref{coste} de la instancia}
     	\STATE{\texttt{result=qap}}
     	\STATE{\texttt{s = qap.size-1}, número de instalaciones}
     	\FORALL{\texttt{k} $\in \{0\ldots s\}$}
	  \STATE{\texttt{fab\_asignada}=[false\ldots false$_s$]}
	  \STATE{\texttt{loc\_asignada}=[false\ldots false$_s$]}
	  \STATE{\texttt{actual=qap}}
	  \STATE{\texttt{i=k}}
	  \FORALL{\texttt{i} $\in \{1\ldots s\}$}
	    \STATE{\texttt{fab\_asignada[i]=true}}
	    \STATE{\texttt{loc\_asignada[actual.permutacion[i]]=true}}
	    \STATE{\texttt{max\_w\_index=j : w(k,j)=}$\max\{\texttt{w(k,j)}\texttt{!fab\_asignada[j]}\}$}
	    \STATE{\texttt{min\_d\_index=j : d(actual.permutacion[k],j)=}
	     \qquad$\min\{\texttt{d(p(k),j)}|\texttt{!loc\_asignada[j]}\}$}
	    \STATE{\texttt{actual.permutacion[max\_w\_index]=min\_d\_index}}
	    \STATE{\texttt{i=max\_w\_index}}
	  \ENDFOR
	  
	  \IF{\texttt{actual.cost < min}}
	    \STATE{\texttt{min=actual.cost}}
	    \STATE{\texttt{result=actual}}
	  \ENDIF  
	\ENDFOR
	\STATE{\texttt{qap=result}}
	\RETURN{qap}
	\end{algorithmic}
    \caption{Heurística greedy$\approx$Vecino más cercano}
    \label{vecino}
\end{algorithm}  
\subsubsection{Análisis de eficiencia}
Las líneas 12 y 13 son \ef(n), y los fragmentos 10-11 y 14-15 son \ef(1)
El bloque formado por las anteriores líneas se ejecuta $n-1$ veces.

Las inicializaciones de las líneas 5-8 son \ef(1), y las líneas 17-20 también.

Luego tenemos un bucle que se ejecuta $n$ veces desde 5 hasta 21, y por tanto
es $n\cdot(n-1)\cdot n = \mathcal{O}(n^3)$

\subsection{Opt-2}
El algoritmo consiste en partir de una permutación, y va tomando todas las posibles
combinaciones de aristas e intercambiándolas. El algoritmo se ejecutará
hasta que no se produzca mejora.
\begin{algorithm}[H]
	\begin{algorithmic}[1]
		\REQUIRE \ \\
        	\texttt{qap}, instancia del problema \\\
     	\STATE{\texttt{continue=true}}
     	\WHILE{\texttt{continue}}
	  \STATE{\texttt{continue}=false}
	  \FORALL{\texttt{i} $\in \{0\ldots (\texttt{qap.size}-1)\}$}
	    \FORALL{\texttt{j} $\in \{(i+1)\ldots (\texttt{qap.size}-1)\}$}
	      \STATE{\texttt{old\_cost=qap.cost}}
	      \STATE{\texttt{qap.permutacion[i]=qap.permutacion[j]}}
	      \STATE{\texttt{qap.permutacion[j]=qap.permutacion[i]}}
	      \IF{\texttt{old\_cost <= qap.cost}}
		\STATE{\texttt{qap.permutacion[i]=qap.permutacion[j]}}
		\STATE{\texttt{qap.permutacion[j]=qap.permutacion[i]}}
	      \ELSE
		\STATE{\texttt{continue=true}}
	      \ENDIF  
	  \ENDFOR
	\ENDFOR
	\ENDWHILE
	\RETURN{qap}
	\end{algorithmic}
    \caption{Heurística 2-OPT}
    \label{opt2}
\end{algorithm}  

\subsubsection{Análisis de eficiencia}
Las líneas 6-13 son \ef(1), y hay dos bucles anidados, que darán eficiencia 
$\sum_{i=1}^{n} \sum_{j=i+1}^{n} 1 = \sum_{i=1}^{n} (n-i)= n^2-\frac{n^2}{2}=\mathcal{O}(n^2)$.

El bloque \texttt{mientras} es $\mathcal{O}(n^2)$. Como sólo sigue el algoritmo si mejora el
coste (menor estricto), 2-OPT no dará dos permutaciones iguales. Como hay $n!$ posibles
permutaciones, el algoritmo realizará como mucho $n!$ etapas, lo que nos asegura que es finito.

\subsubsection{Elección de la permutación inicial}
Para realizar OPT-2, podríamos partir de una permutación inicial no trivial,
que podríamos generar con cualquiera de las dos heurísticas greedy que se presentan
en este documento. En ese caso, la eficiencia de OPT-2 sería:
$$\mathcal{O}(\max(n^2,\texttt{eficiencia(greedy)}))$$

\subsection{Heurística greedy v2}
La idea de esta segunda heurística es colocar la instalación aún no asignada que tiene 
mayor suma de flujos con el resto de instalaciones, en la localización más
céntrica aún no asignada, esto es la que tenga menor suma de distancias al resto de 
localizaciones.

\begin{algorithm}[H]
	\begin{algorithmic}[1]
		\REQUIRE \ \\
        	\texttt{qap}, instancia del problema \\\
	\STATE{\texttt{fab\_asignada}=[false\ldots false$_s$]}
	\STATE{\texttt{loc\_asignada}=[false\ldots false$_s$]}
	\STATE{\texttt{w=[]}}
     	\STATE{\texttt{d=[]}}
     	\STATE{\texttt{s = qap.size-1}, número de instalaciones}
     	\FORALL{\texttt{i} $\in \{0\ldots s\}$}
	  \STATE{\texttt{w[i]=}$\sum_{j=0}^s\texttt{qap.weight(i,j)}$}
	  \STATE{\texttt{d[i]=}$\sum_{j=0}^s\texttt{qap.distance(i,j)}$}
	\ENDFOR
	\FOR{\texttt{k=0;k<=s;k++}}
	  \STATE{\texttt{max\_w\_index=j: w[j]=}$\max\{w[i]:\texttt{!fab\_asignada[i]}\}$}
	  \STATE{\texttt{min\_d\_index=j: d[j]=}$\min\{d[i]:\texttt{!loc\_asignada[i]}\}$}
	  \STATE{\texttt{qap.permutacion[max\_w\_index]=min\_d\_index}}
	  \STATE{\texttt{fab\_asignada[max\_w\_index]=true}}
	  \STATE{\texttt{loc\_asignada[max\_w\_index]=true}}
	\ENDFOR
	\RETURN{qap}
	\end{algorithmic}
    \caption{Heurística greedy$\approx$Vecino más cercano}
    \label{greedy2}
\end{algorithm}  
\subsubsection{Análisis de eficiencia}
Las inicializaciones de las líneas 5,6 son \ef(1), y el for 4-7, \ef(n).

Las líneas 11-13 son \ef(1), y la 9 y la 10 \ef(n) cada una. Están en un bucle desde
las líneas 8-14 que se ejecuta n veces, y que por tanto es $\mathcal{O}(n^2)$.

Luego la eficiencia del segundo greedy es $\mathcal{O}(\max(n,n^2))=\mathcal{O}(n^2)$ 

\section{Comparativa de heurísticas}
Se ha realizado la ejecución de las cuatro heurísticas con los distintos archivos de datos. En la siguiente tabla mostramos los costes obtenidos calculados mediante la función de coste de la ecuación \ref{coste}.

Se puede observar que la mejor heurística es 2-opt, aunque es mucho menos eficiente como se ha visto antes. En general la heurística Greedy basada en \textit{Vecino más cercano} consigue en mejores resultados que la Greedy-v2, exceptuando algunos casos como \texttt{tai100b}, \texttt{tai150b}, \texttt{tho150}, \texttt{wil100}.

\begin{center}
    \begin{longtabu} to \linewidth{ l | *4{d{0}}}  % máx 10 decimales
\rowfont\bfseries Datos & \multicolumn{1}{l}{inicial}& \multicolumn{1}{l}{greedy-v1}& \multicolumn{1}{l}{greedy-v2}& \multicolumn{1}{l}{2-opt}\\ \hline
    \endhead
    \endfoot
    \\ \hline
    \endlastfoot
bur26a.dat & 5801101 & 5708893 & 5968117 & 5445082 \\
bur26b.dat & 4127073 & 4036424 & 4323853 & 3843592 \\
bur26c.dat & 5902338 & 5656736 & 6029099 & 5441766 \\
bur26d.dat & 4208557 & 4059566 & 4301307 & 3871912 \\
bur26e.dat & 5957294 & 5689754 & 6149822 & 5398777 \\
bur26f.dat & 4250781 & 4075045 & 4451226 & 3797323 \\
bur26g.dat & 11034476 & 10698666 & 11407304 & 10141454 \\
bur26h.dat & 7877391 & 7632171 & 7999869 & 7102234 \\
chr12a.dat & 40172 & 27660 & 41064 & 15034 \\
chr12b.dat & 40768 & 25800 & 38320 & 10766 \\
chr12c.dat & 25162 & 24330 & 47886 & 17388 \\
chr15a.dat & 78700 & 34734 & 59358 & 15502 \\
chr15b.dat & 61200 & 22772 & 47830 & 10924 \\
chr15c.dat & 58994 & 44784 & 60030 & 17934 \\
chr18a.dat & 72342 & 52104 & 54426 & 22350 \\
chr18b.dat & 2926 & 2926 & 4856 & 1712 \\
chr20a.dat & 10478 & 7626 & 8100 & 3418 \\
chr20b.dat & 6598 & 6598 & 10346 & 2808 \\
chr20c.dat & 122026 & 74628 & 77766 & 19632 \\
chr22a.dat & 12072 & 10514 & 12138 & 6892 \\
chr22b.dat & 15532 & 10700 & 13280 & 6908 \\
chr25a.dat & 19750 & 14484 & 20414 & 6094 \\
lipa20a.dat & 3958 & 3895 & 3978 & 3796 \\
lipa20b.dat & 27076 & 27076 & 34587 & 27076 \\
lipa30a.dat & 13931 & 13696 & 13765 & 13451 \\
lipa30b.dat & 151426 & 151426 & 190904 & 151426 \\
lipa40a.dat & 32770 & 32532 & 32837 & 31998 \\
lipa40b.dat & 476581 & 476581 & 596219 & 476581 \\
lipa50a.dat & 64142 & 63768 & 64109 & 62811 \\
lipa50b.dat & 1210244 & 1210244 & 1543450 & 1210244 \\
lipa60a.dat & 110220 & 109742 & 110296 & 108325 \\
lipa60b.dat & 2520135 & 2520135 & 3218450 & 2520135 \\
lipa70a.dat & 173736 & 173384 & 173783 & 171325 \\
lipa70b.dat & 4603200 & 4603200 & 5936240 & 4603200 \\
lipa80a.dat & 258846 & 257735 & 258204 & 255369 \\
lipa80b.dat & 7763962 & 7763962 & 10037083 & 7763962 \\
lipa90a.dat & 367478 & 366860 & 367742 & 363266 \\
lipa90b.dat & 12490441 & 12490441 & 16165958 & 12490441 \\
nug12.dat & 724 & 670 & 814 & 606 \\
nug14.dat & 1298 & 1164 & 1256 & 1048 \\
nug15.dat & 1492 & 1318 & 1444 & 1206 \\
nug16a.dat & 2156 & 1946 & 1918 & 1688 \\
nug16b.dat & 1676 & 1460 & 1512 & 1286 \\
nug17.dat & 2350 & 2090 & 1938 & 1812 \\
nug18.dat & 2586 & 2306 & 2318 & 1982 \\
nug20.dat & 3444 & 3106 & 3148 & 2666 \\
nug21.dat & 3474 & 2968 & 3130 & 2514 \\
nug22.dat & 5030 & 4482 & 4584 & 3720 \\
nug24.dat & 4622 & 4278 & 4328 & 3636 \\
nug25.dat & 4838 & 4566 & 4438 & 3946 \\
nug27.dat & 7392 & 6464 & 6506 & 5442 \\
nug28.dat & 7010 & 6322 & 6348 & 5342 \\
nug30.dat & 8060 & 7444 & 7516 & 6388 \\
tai100a.dat & 23984176 & 23618208 & 23936546 & 21811806 \\
tai100b.dat & 1782212399 & 1709267903 & 1574846615 & 1242475763 \\
tai150b.dat & 653551032 & 631832184 & 623469733 & 518835878 \\
% tai256c.dat \\
tai60a.dat & 8469610 & 8130436 & 8373936 & 7527292 \\
tai60b.dat & 1027374245 & 939190011 & 957301714 & 633106011 \\
tai64c.dat & 5893540 & 5530474 & 5893540 & 1860942 \\
tai80a.dat & 15688718 & 15324492 & 15638632 & 14026150 \\
tai80b.dat & 1260924801 & 1169112773 & 1200056944 & 843497614 \\
tho150.dat & 9842324 & 9567796 & 9527466 & 8269216 \\
wil100.dat & 299832 & 294312 & 293152 & 276198
    \end{longtabu}
\end{center}

\end{document}